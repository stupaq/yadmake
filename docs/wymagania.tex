%% IOP - wymagania projektu
%

\documentclass[a4paper]{article}

\usepackage[utf8]{inputenc}
\usepackage{polski}
\usepackage{fullpage}
\usepackage{graphicx}

\title{Wymagania projektu ,,System rozproszonej kompilacji''}
\author{Marta Drozdek, Anna Lewicka, Wacław Banasik, Mateusz Machalica}
\date{\today}

\begin{document}

\maketitle

\begin{table}[!h]
	\centering
	\begin{tabular}{|c|c|c|c|}
		\hline
		\textbf{Data} & \textbf{Wersja} & \textbf{Opis zmian} & \textbf{Autorzy} \\ \hline
		19/03/2012 & 1 & Wstępna wersja dokumentu & Cały zespół \\ \hline
		25/05/2012 & 2 & Aktualizacja scenariuszy alternatywnych i listy implementowanych opcji & Mateusz Machalica \\ \hline
	\end{tabular}
\end{table}

\section{Założenia wstępne}

System wymaga, aby na każdej maszynie, zarówno zdalnej jak i jednostce zarządzającej procesem konfiguracji był zainstalowany i skonfigurowany interpreter poleceń bash, daemon ssh oraz wszystkie programy, które byłyby potrzebne to lokalnego zbudowania projektu przy pomocy programu GNUMake. Dodatkowo na maszynie, na której jest wykonywany program, powinien być poprawnie zainstalowany GNUMake i lokalizacja jego pliku wykonywalnego powinna być dodana do zmiennej środowiskowej \verb+$PATH+.

\section{Wymagania funkcjonalne}

Podstawowe wymagania funkcjonalne dotyczące produktu wynikają z jego docelowego zastosowania jako zamiennika GNUMake’a przy budowaniu złożonych projektów. Wymagania funkcjonalne zostały wyspecyfikowane poprzez przypadki użycia. Aktorzy występujący w scenariuszach to:
\begin{itemize}
	\item użytkownik - osoba chcąca skompilować program przy użyciu systemu
	\item centralny komputer na którym uruchamiana jest kompilacja
	\item pomocnicze jednostki kompilacji dostępne w sieci
\end{itemize}

\subsection{Scenariusz standardowy}

\begin{enumerate}
	\item Użytkownik podaje, jakie komputery są do dyspozycji
	\item Użytkownik rozpoczyna kompilację wpisując w konsolę polecenie \verb+yadmake+ z argumentami jak w GNUMake (cele, [opcje]) 
	\item System buduje podane cele
	\item Efekt jest taki, jak przy użyciu GNUMake (wypisuje standardowe wyjście i wyjście błędów wykonywanych komend, zwraca odpowiedni kod wyjścia), pliki wynikowe znajdują się w katalogu, w którym został wywołany program
\end{enumerate}


\subsection{Scenariusz z błędem 1}

\begin{enumerate}
	\item Nie udaje się nawiązać połączenia z niektórymi komputerami
	\item System zgłasza komunikat o błędzie połączenia
	\item Program działa jak w scenariuszu standardowym na dostępnych komputerach
\end{enumerate}


\subsubsection{Scenariusz z błędem 1.1}

\begin{enumerate}
	\item Nie udaje się nawiązać żadnego połączenia
	\item System zgłasza komunikat o błędzie połączenia
	\item Kompilacją zostaje porzucona
\end{enumerate}


\subsection{Scenariusz z błędem 2}

\begin{enumerate}
	\item Błąd składni w Makefile’u
	\item Wypisanie komunikatu błędu, dokładnie w takim formacie w jakim wypisałby go GNUMake
\end{enumerate}


\subsection{Scenariusz z błędem 3}

\begin{enumerate}
	\item Błąd kompilacji na jednym z komputerów
	\item Dokończenie tworzenia równolegle kompilowanych celów na pozostałych jednostkach
	\item Wypisanie komunikatu o błędzie i kodu zwróconego przez polecenie, które spowodowało ten błąd
\end{enumerate}


\subsection{Scenariusz z błędem 4}

\begin{enumerate}
	\item Nierealizowalny plan wykonania kompilacji
		\begin{itemize}
			\item Cykl w zależnościach
			\item Brak plików źródłowych
			\item Brak przepisu na zbudowanie jednego z plików pośrednich lub samego celu
		\end{itemize}
	\item Komunikat o błędzie z podaniem przyczyny
\end{enumerate}

\subsection{Opcje implementowane przez program}

Program będzie implementował podzbiór opcji programu GNUMake oraz opcje sterujące procesem kompilacji. Wszystkie opcje mają być zgodne z formatem \verb+getopts+ (w wersji POSIX, a zatem bez długich opcji).
Implementowane opcje:

\begin{itemize}
	\item \verb+-k+ -- nie kończy procesu budowania przy pierwszym błędzie, ale kontynuuje kompilację celów, na których kompilację pozwalają wykonane już zależności
	\item \verb+-B+ -- nie uwzględnia czasów modyfikacji celów i plików źródłowych, traktując wszystkie jako zmodyfikowane
	% TODO implementowane opcje 
\end{itemize}


\section{Wymagania pozafunkcjonalne}

\subsection{Wygoda}

Jako zamiennik GNUMake’a program powinien obsługiwać taką samą składnię jak oryginał i udostępniać taki sam interfejs użytkownika, format komunikatów wypisywanych przez program będzie zachowywał zgodność z pierwowzorem tam, gdzie to możliwe, z racji rozszerzonej funkcjonalności.
Jedyną wspieraną wersją językową będzie wersja angielska.


\subsection{Bezpieczeństwo}

W zależności od konfiguracji, można opisywać dwa przypadki użycia systemu, które istotnie różnią się wymaganiami odnośnie bezpieczeństwa.

\subsubsection{Zaufana sieć (lokalna)}

W przypadku wykorzystania systemu do budowy oprogramowania na zdalnych maszynach, znajdujących się w zaufanym środowisku, np. tej samej sieci lokalnej, nie wymaga się szyfrowania połączeń pomiędzy komputerem zlecającym kompilację a zdalnymi maszynami.

\subsubsection{Niezaufane medium pośredniczące}

W przypadku wykorzystania do komunikacji pomiędzy maszynami np. Internetu, wymagane jest pełne szyfrowanie, protokół zdalnego wykonywania zadań oraz wymiany plików źródłowych i wynikowych powinien zapewniać przynajmniej taką poufność i potwierdzenie tożsamości, jak protokół SSHv2.


\subsection{Szybkość}

Docelowo system ma przyspieszać kompilację projektów, które składają się z wielu niezależnych części i których kompilacja może zostać zrównoleglona w wystarczającym stopniu aby zniwelować opóźnienia spowodowane zdalną komunikacją maszyn uczestniczących w procesie kompilacji.
Jeśli zaniedbać opóźnienia związane ze zdalną komunikacja przez sieć, system ma działać nie wolniej niż wykonanie GNUMake’a na pojedynczym komputerze.


\subsection{Usability}

Z racji zamierzonej zgodności ze standardem (jakim są pochodne make’a) w większości wypadków nie będą potrzebne żadne modyfikacje do istniejących skryptów, obsługa programu nie będzie również wymagała zaznajamiania się z nowymi opcjami, zamierzamy implementować podzbiór opcji GNUMake’a i dodatkowo flagi regulujące działanie systemu koordynowania zadań na zdalnych maszynach.


\subsection{Komunikaty o błędach}

Komunikaty o błędach mają być takie same jak w programie GNUMake, dodatkowo komunikaty o niepoprawnej konfiguracji maszyn kompilujących, braku łączności lub innych problemach związanych ze zdalnym wykonywaniem zadań będą wypisywanie w terminalu, w którym zostanie wykonany program.
Tak jak GNUMake, program będzie wypisywał standardowe wyjście i wyjście błędów wszystkich komend, które wykona celem zbudowania zawartości opisanej w Makefile’u.


\subsection{Estetyka}

Z racji sposobu interakcji z użytkownikiem ten punkt nie stosuje się do opisywanego systemu.


\end{document}
