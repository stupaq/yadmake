%% IOP - plan zarządzania konfiguracją
%

\documentclass[a4paper]{article}

\usepackage{polski}
\usepackage[utf8]{inputenc}
\usepackage{fullpage}
\usepackage{indentfirst}
\usepackage{graphicx}
\usepackage{ifthen}
\usepackage{amssymb}
\usepackage{amsthm}
\usepackage{wrapfig}
\usepackage{siunitx}
\usepackage{xstring}
\usepackage{float}


\title{Plan zarządzania konfiguracją projektu ,,System rozproszonej kompilacji''}
\author{Marta Drozdek, Anna Lewicka, Wacław Banasik, Mateusz Machalica}
\date{\today}

\begin{document}

\maketitle

\begin{table}[!h]
	\centering
	\begin{tabular}{|c|c|p{3cm}|p{9cm}|}
		\hline
		\textbf{Data} & \textbf{Wersja} & \textbf{Autorzy} & \textbf{Opis zmian} \\ \hline
		01/04/2012 & 1 & Anna Lewicka, Marta Drozdek & Wstępna wersja dokumentu \\ \hline
		29/05/2012 & 2 & Mateusz Machalica & Dodanie wstępu i formalizacja użytych zwrotów i pojęć, formalne określenie butów podlegających zarządzaniu, i sposobu obsługi bazy danych konfiguracji oraz systemu zarządzania projektem. Zaktualizowano schemat wydań. \\ \hline
	\end{tabular}
\end{table}

\section{Wstęp}

Celem tego dokumentu jest opisanie przyjętych metod zarządzania konfiguracją, to znaczy procedur i standardów podejmowanych celem panowania nad ewolucją projektu.

\section{Byty podlegające zarządzaniu konfiguracjami}

Zarządzaniu konfiguracją podlegają:
\begin{itemize}
	\item kod źródłowy projektu
	\item dokumentacja potrzebna przy dalszym rozwoju projektu: wizja oraz architektura projektu, plany zarządzania konfiguracją i jakością oraz opis wymagań projektu
	\item wytworzone narzędzia oraz skrypty do budowy, konfigurowania, instalacji i testowania systemu
	\item wyniki testów wydajnościowych (w późniejszy, stabilnych wydaniach systemu)
	\item pliki źródłowe, przy pomocy których automatycznie generowana jest dokumentacja kodu źródłowego
	\item zgłoszenia usterek, żądania zmiany lub rozbudowy funkcjonalności (wraz z aktualnym stanem zgłoszenia, to znaczy opisem kto jest odpowiedzialny za wprowadzenie tej zmiany i w jakim stopniu została wprowadzona w najnowszej wersji)
	\item raporty stwierdzające wykonanie lub odrzucenie żądanej zmiany w projekcie lub naprawy usterki
\end{itemize}

\section{Baza danych konfiguracji}

Konfiguracje projektu (byty podlegające zarządzaniu) przechowywane są w repozytorium projektu w serwisie GitHub poza żądaniami zmian lub naprawy usterek i raportami z postępu tych prac, które znajdują się w systemie zarządzania projektem serwisu GitHub\footnote{zakładka ,,Issues'' na głównej stronie projektu}.

\subsection{Organizacja repozytorium}
 
Każdy moduł rozwijany jest w osobnej gałęzi. W oddzielnych gałęziach przechowujemy testy do poszczególnych modułów.

W gałęzi \verb+master+ przechowujemy wersje przetestowane, implementujące kompletną funkcjonalność przewidywaną dla danej wersji oprogramowania, oznaczone etykietą (,,tag'') zawierającą numer wersji, w formacie opisanym w dalszej części dokumentu.

Oprócz tego w oddzielnej gałęzi \verb+docs+ przechowujemy dokumentację projektu.

Z racji rozpoczęcia użycia bazy konfiguracji przed ustaleniem standardów odnośnie opisu commit'ów zapisywanych w bazie, dopuszczamy aby do ukończenia prac nad wersją ,,0.1'' używać w opisach języka polskiego lub angielskiego.

\subsection{Informacje o wersjach dla zespołu}

Każdy moduł powinien  być opatrzony komentarzem opisującym status wersji przed włączeniem do gałęzi \verb+master+. Powinien on informować, czy dana wersja została przetestowana i czy pliki źródłowe zostały zaakceptowane przez innego członka zespołu niż ich autor.

\section{Identyfikacja wersji}

\subsection{Nazewnictwo i numerowanie}

Kolejne wersje oprócz nazwy programu będą oznaczane wersją -- dwoma liczbami naturalnymi oddzielonymi kropką. Pierwszą wersją będzie ,,0.1''.
Zwiększenie pierwszej liczby będzie oznaczało istotne zmiany funkcjonalności, zaś zwiększenie drugiej -- wprowadzenie poprawek nie wpływających na zakres funkcjonalności systemu, ale potencjalnie powodujących poprawę wydajności lub stabilności oprogramowania.
Wersje stabilne będą oznaczone dodatkowo słowem ,,stable'' znajdującym się po numerze wersji.

\subsection{Informacja o wersjach dla użytkownika}

Wywołując program z opcją \verb+-v+ użytkownik będzie mógł dowiedzieć się, z jakiej wersji programu aktualnie korzysta. Dane na temat wersji będą trzymane w jednym z plików źródłowych w formie tablicy znaków zakończonej terminalnym zerem. 

\section{Zarządzanie wydaniami}

Przyjmujemy zasadę, że dla każdej wersji stabilnej oprogramowania tworzymy nowe wydanie.
Z racji niekomercyjnego i eksperymentalnego charakteru projektu nie jesteśmy ograniczeni takimi czynnikami, jak propozycje zmian od klienta czy wymagania rynkowo-marketingowe.

Każde wydanie będzie zawierało pełną dokumentację projektu (wraz z dokumentacją kodu, generowaną automatycznie), skrypty potrzebne do budowy systemu oraz przykładowe pliki konfiguracyjne.
Nie zamierzamy dostarczać binarnych paczek.

\section {Kontrola stanu projektu zarządzanie zmianami}

Każdej wersji systemu odpowiada kamień milowy w systemie zarządzania projektem o nazwie takiej jak numer wersji.

Wszystkie usterki, pytania i żądania zgłaszane w systemie zarządzania projektem w serwisie GitHub, jeśli dotyczą konkretnej wersji powinny być przypisane do kamienia milowego odpowiadającego tej wersji.

Treść zgłoszenia powinna zawierać określenie żądanych zmian (poprawy lub rozszerzenia funkcjonalności, zmiany implementacji i innych).
Jeśli została wyznaczona osoba, która jest odpowiedzialna za rozwiązanie zgłoszonego problemu, to informacja o tym powinna znaleźć się w systemie w postaci przypisania danego problemu do tej osoby.
System komentarzy pozwala wyrazić opinie i notować zalety, wady oraz inne spostrzeżenia, jak na przykład koszt wprowadzenia zmiany i przewidywane zyski.
Jeśli zgłoszenie zostanie zamknięte to w komentarzu powinien się pojawić opis decyzji podjętych odnośnie zmiany oraz w przypadku, gdy nie panowała całkowita zgoda odnośnie podjętej decyzji lub decyzja została podjęta dzięki nietrywialnej obserwacji (na przykład wynikom testów), w komentarzu należy umieścić również opis czynników, które wpłynęły na tą decyzję (jej uzasadnienie).

Zgłoszenia mogą zostać zamknięte dopiero po poprawieniu błędu, dodaniu funkcjonalności, podjęciu decyzji projektowej odpowiadającej na pytanie lub odrzuceniu zgłoszenia.

\subsection{Rozszerzenie funkcjonalności}

Propozycja rozszerzenia funkcjonalności zgłaszana jest w środowisku GitHub z etykietą ,,enhancement''.
Członkowie zespołu wspólnie omawiają sensowność propozycji, następnie jest ona przyjmowana bądź odrzucana przez zespół.
Po dokonaniu zmian nowa wersja jest testowana. Jeśli okaże się lepsza od poprzedniej, to zostaje włączona do systemu.

\subsection{Zgłaszanie błędów}

Zauważone błędy w bieżącej wersji są zgłaszane w środowisku GitHub z etykietą ,,bug''.
Autor fragmentu kodu, w którym znaleziono błąd jest odpowiedzialny za poprawienie go i zamknięcie informacji o błędzie wraz z opisem działań podjętych celem poprawienia błędu.

\subsection{Zgłaszanie prośby o analizę decyzji projektowej}

W systemie zarządzania projektem należy również zgłaszać z etykietą ,,question'' wszystkie wynikłe w trakcie implementacji lub projektowania konkretnych modułów wątpliwości, nie opisane w dokumentach takich jak plany zarządzania konfiguracją i jakością, opisie wymagań lub architektury.
Należy przy tym zgłaszać jedynie wątpliwości, które mają pływ na inne moduły lub architekturę systemu, nie wątpliwości implementacyjne, jeśli nie kwalifikują się jako jedne z poprzednich.

Dla usystematyzowania, za decyzje projektowe (a więc odpowiadające na wątpliwości/pytania, które należy zgłaszać w systemie zarządzania projektem) uznajemy decyzje:
\begin{itemize}
	\item wpływające na inne moduły oprogramowania lub architekturę systemu
	\item wpływające na wydajność systemu lub spełnialność innych wymagań pozafunkcjonalnych, jeśli nie są natury implementacyjnej
	\item wymagające dużego nakładu pracy potrzebnej do wprowadzenia zmian
\end{itemize}

Podjęte decyzje powinny zostać opisane wraz z uzasadnieniem w komentarzu pod danym pytaniem (po podjęciu decyzji zgłoszenie może zostać zamknięte).

Podjęta decyzja nie jest zgodna z treścią dokumentacji projektu, to przed zamknięciem zgłoszenia dokumentacja musi zostać zaktualizowana.

\end{document}
