%% Plan zarządzania konfiguracją
%

\documentclass[a4paper]{article}

\usepackage{polski}
\usepackage[utf8]{inputenc}
\usepackage{fullpage}
\usepackage{indentfirst}
\usepackage{graphicx}
\usepackage{ifthen}
\usepackage{amssymb}
\usepackage{amsthm}
\usepackage{wrapfig}
\usepackage{siunitx}
\usepackage{xstring}
\usepackage{float}


\title{Plan zarządzania konfiguracją projektu ,,System rozproszonej kompilacji''}
\author{Marta Drozdek, Anna Lewicka, Wacław Banasik, Mateusz Machalica}
\date{01 kwietnia 2012~r.}

\begin{document}

\maketitle

\begin{table}[!h]
	\centering
	\begin{tabular}{|c|c|c|c|}
		\hline
		\textbf{Data} & \textbf{Wersja} & \textbf{Opis zmian} & \textbf{Autorzy} \\ \hline
		01/04/2012 & 1 & Wstępna wersja dokumentu & Anna Lewicka, Marta Drozdek \\ \hline
	\end{tabular}
\end{table}

\section{Przechowywanie wersji}

\subsection{Organizacja repozytorium}
 
Pliki i dokumenty trzymane są w repozytorium GitHub. Każdy moduł rozwijamy w osobnej gałęzi. W oddzielnych gałęziach przechowujemy testy do poszczególnych modułów. W gałęzi master przechowujemy wersje przetestowane, gotowe do wydania. Oprócz tego w oddzielnej gałęzi przechowujemy dokumentację projektu.

\subsection{Informacje o wersjach dla zespołu}

Każdy moduł powinien  być opatrzony komentarzem opisującym status wersji. Powinien on informować, czy dana wersja została przetestowana i czy pliki źródłowe zostały zaakceptowane przez innego członka zespołu niż ich autor.

\section{Identyfikacja wersji}

\subsection{Nazewnictwo i numerowanie}

Kolejne wersje oprócz nazwy programu będą oznaczane wersją -- dwoma liczbami naturalnymi oddzielonymi kropką. Pierwszą wersją będzie ,,1.0''. Zwiększenie pierwszej liczby będzie oznaczało rozszerzenie lub zmianę funkjonalności, zaś zwiększenie drugiej -- poprawienie poprzedniej wersji.

\subsection{Informacja o wersjach dla użytkownika}

Wywołując program z opcją \verb+-v+ użytkownik będzie mógł dowiedzieć się, z jakiej wersji programu aktualnie korzysta. Dane na temat wersji będą trzymane w jednym z plików źródłowych. 

\section {Kontrola konfiguracji}

\subsection{Zmiana wersji}

Propozycja zmiany wersji zgłaszana jest w środowisku GitHub z etykietą ,,enhancement'' (ang. ulepszenie, poprawa). Członkowie zespołu wspólnie omawiają sensowność propozycji, następnie jest ona przyjmowana bądź odrzucana przez zespół. Po dokonaniu zmian nowa wersja jest testowana. Jeśli okaże się lepsza od poprzedniej, to zostaje wydana z odpowiednim numerem.

\subsection{Zgłaszanie błędów}

Zauważone błędy w bieżącej wersji są zgłaszane w środowisku GitHub z etykietą ,,bug''. Autor fragmentu kodu, w którym znaleziono błąd jest odpowiedzialny za poprawienie go i zamknięcie informacji o błędzie.

\end{document}
