%% IOP - wymagania projektu
%

\documentclass[a4paper]{article}

\usepackage[utf8]{inputenc}
\usepackage{polski}
\usepackage{fullpage}
\usepackage{graphicx}

\title{Wymagania projektu ,,System rozproszonej kompilacji''}
\author{Marta Drozdek, Anna Lewicka, Wacław Banasik, Mateusz Machalica}
\date{\today}

\begin{document}

\maketitle

\begin{table}[!h]
	\centering
	\begin{tabular}{|c|c|c|p{8cm}|}
		\hline
		\textbf{Data} & \textbf{Wersja} & \textbf{Autorzy} & \textbf{Opis zmian} \\ \hline
		19/03/2012 & 1 & Cały zespół & Wstępna wersja dokumentu \\ \hline
		25/05/2012 & 2 & Mateusz Machalica & Aktualizacja scenariuszy alternatywnych, listy implementowanych opcji i wymagań odnośnie konfiguracji systemu \\ \hline
		27/05/2012 & 3 & Mateusz Machalica & Dodano wstęp i przepisano dokument formalizując pojęcia oraz sam opis przypadków użycia \\ \hline
	\end{tabular}
\end{table}

\section{Wstęp}

Celem tego dokumentu jest opis wymagań funkcjonalnych i pozafunkcjonalnych dotyczących działania gotowego produktu.
Nie specyfikujemy tutaj wymagań odnośnie kodu źródłowego, dokumentacji i procesu weryfikacji działania produktu. Wymagania te są opisane w ,,Planie zarządzania konfiguracją'' oraz ,,Planie zarządzania jakością'' projektu.

\input{slownik_pojec.tex.inc}

\section{Założenia wstępne}

System wymaga, aby na każdej maszynie, zarówno zdalnej jak i jednostce koordynującej procesem konfiguracji był zainstalowany i skonfigurowany interpreter poleceń Bash, serwer i klient SSH oraz wszystkie programy, które byłyby potrzebne to lokalnego zbudowania projektu przy pomocy programu GNUMake.
Dodatkowo na maszynie, na której jest wykonywany program, powinien być poprawnie zainstalowany GNUMake i lokalizacja jego pliku wykonywalnego powinna być dodana do zmiennej środowiskowej \verb+$PATH+.

System plików na wszystkich maszynach uczestniczących w procesie kompilacji powinien obsługiwać zapisywanie czasów modyfikacji plików i zegary na tych maszynach powinny być zsynchronizowane, aby zapewnić poprawne uwzględnianie przez program, które cele zostały już wykonane, a które należy jeszcze wykonać.
W przypadku, gdy powyższe warunki nie są spełnione program nie jest w stanie zapewnić, że podczas kompilacji nie będzie ponownie wykonywał już aktualnych celów.

\section{Wymagania funkcjonalne}

Podstawowe wymagania funkcjonalne dotyczące programu wynikają z jego docelowego zastosowania jako zamiennika programu GNUMake przy budowaniu złożonych projektów. Wymagania funkcjonalne zostały wyspecyfikowane poprzez przypadki użycia. Aktorzy występujący w scenariuszach to:
\begin{itemize}
	\item \emph{Użytkownik} -- osoba chcąca skompilować program przy użyciu programu
	\item \emph{Jednostka koordynująca} -- centralny komputer na którym uruchamiana jest kompilacja i na który mają trafić wyniki działania programu
	\item \emph{Jednostki pomocnicze} -- komputery, które pozostają do dyspozycji \emph{jednostki koordynującej} i na których można zdalnie wykonywać polecenia
\end{itemize}

Dla odróżnienia nazwy aktorów będą dalej pisane kursywą.

\subsection{Scenariusze i przypadki użycia}

Standardowy sposób użycia programu polega na uruchomieniu go w katalogu, w którym znajdują się pliki źródłowe potrzebne do wykonania opisu procesu kompilacji zawartego w pliku Makefile (również znajdującego się w katalogu). Przed uruchomieniem \emph{użytkownik} powinien skonfigurować program zgodnie z przypadkiem użycia opisanym poniżej.

\subsubsection{Konfiguracja systemu}

\begin{enumerate}
	\item \emph{Użytkownik} wprowadza w pliku \verb+$PAHT/.yadmake/hosts+ nazwy hostów będących \emph{jednostkami pomocniczymi}.
	\item \emph{Użytkownik} wprowadza w pliku \verb+$PAHT/.ssh/config+ parametry potrzebne do zdalnego połączenia przy pomocy SSH z hostami wprowadzonymi w poprzednim kroku. Format tego pliku jest niezależny od systemu rozproszonej kompilacji, stanowi część pakietu SSH.
\end{enumerate}

\subsubsection{Użycie systemu -- scenariusz podstawowy}

\begin{enumerate}
	\item \emph{Użytkownik} rozpoczyna proces kompilacji przechodząc do katalogu zawierającego plik Makefile z opisem procesu kompilacji i wywołując program z opcjonalnymi argumentami opisanymi w dalszej części wymagań.
	\item Program wykonuje polecenia potrzebne do wykonania domyślnych celów opisanych w pliku Makefile. Jeśli dowolny cel pośredni istnieje w systemie plików nie jest potrzebna jego aktualizacja, to znaczy jego data aktualizacji jest późniejsza niż daty aktualizacji wszystkich celów lub plików, od których zależy, to nie jest ponownie budowany.
	\item Efekt jest taki, jak przy użyciu programu GNUMake. Zostają zbudowane wszystkie cele domyślne i dodatkowo wszystkie te cele, które według specyfikacji były potrzebne do zbudowania celi domyślnych, w trakcie kompilacji program wypisuje na standardowe wyjście wszystko co na standardowe wyjście wypisały programy wywoływane podczas procesu kompilacji, pliki wynikowe znajdują się w katalogu, w którym został wywołany program.
	\item Program kończy z kodem wyjścia równym zero.
\end{enumerate}

\subsubsection{Użycie systemu -- scenariusz alternatywny 1.}

\begin{enumerate}
	\item Nie udaje się nawiązać połączenia z niektórymi, ale nie wszystkimi z \emph{jednostek pomocniczych}.
	\item System zgłasza komunikat o błędzie połączenia.
	\item Program działa jak w scenariuszu standardowym na dostępnych \emph{jednostkach pomocniczych}.
\end{enumerate}

\subsubsection{Użycie systemu -- scenariusz alternatywny 2.}

\begin{enumerate}
	\item Nie udaje się nawiązać połączenia z żadną z \emph{jednostek pomocniczych}.
	\item System zgłasza komunikat o błędzie połączenia.
	\item Kompilacją zostaje porzucona.
	\item Program kończy z kodem wyjścia równym jeden.
\end{enumerate}

\subsubsection{Użycie systemu -- scenariusz alternatywny 3.}

\begin{enumerate}
	\item Błąd składni w pliku Makefile.
	\item Program wypisuje komunikatu błędu, zawierający opis błędu i wyjście programu GNUMake uruchomionego dla takiego pliku Makefile.
	\item Kompilacją zostaje porzucona.
	\item Program kończy z kodem wyjścia równym jeden.
\end{enumerate}

\subsubsection{Użycie systemu -- scenariusz alternatywny 4.}

\begin{enumerate}
	\item Błąd kompilacji na jednej z \emph{jednostek pomocniczych}. Przyczyna błędu może być dowolna, jednak nie związana z problemami z łącznością z danym komputerem. Dopuszczamy zatem w tym przypadku użycia jedynie błędy, które wystąpiłyby również przy lokalnym wykonywani kompilacji przy pomocy programu GNUMake.
	\item Dokończenie tworzenia równolegle kompilowanych celów na pozostałych \emph{jednostkach pomocniczych}.
	\item Wypisanie komunikatu o błędzie i kodu wyjścia programu, który spowodował ten błąd.
	\item Program kończy z kodem wyjścia równym dwa.
\end{enumerate}

\subsubsection{Użycie systemu -- scenariusz alternatywny 5.}

\begin{enumerate}
	\item Błąd połączenia -- zerwanie już istniejącego połączenia lub niemożność wykonania polecenia na \emph{jednostce pomocniczej} spowodowana problemami ze zdalną komunikacją.
	\item Kontynuacja procesu kompilacji jak w scenariuszu standardowym.
\end{enumerate}

\subsubsection{Użycie systemu -- scenariusz alternatywny 6.}

\begin{enumerate}
	\item Plan wykonania kompilacji, który nie jest możliwy do wykonania, z powodu cyklu w zależnościach.
	\item Program zachowuje się jak GNUMake -- jeśli choć jeden cel w cyklu musi być przebudowany, to wszystkie cele w cyklu będą przebudowane.
	\item Kontynuacja procesu kompilacji jak w scenariuszu standardowym.
\end{enumerate}

\subsection{Opcje implementowane przez program}

Program będzie implementował podzbiór opcji programu GNUMake oraz dodatkowe opcje sterujące procesem kompilacji na zdalnych maszynach.
Wszystkie opcje mają być zgodne z formatem \verb+getopts+ (w wersji POSIX, a zatem bez długich opcji w stylu GNU).
Implementowane opcje:

\begin{itemize}
	\item \verb+-k+ -- nie kończy procesu budowania przy pierwszym błędzie, ale kontynuuje kompilację celów, na których kompilację pozwalają wykonane już zależności
	\item \verb+-B+ -- nie uwzględnia czasów modyfikacji celów i plików źródłowych, traktując wszystkie jako zmodyfikowane
	\item \verb+-v+ -- program wypisuje wersję programu i natychmiast kończy działanie
	%TODO implementowane opcje 
\end{itemize}

\section{Wymagania pozafunkcjonalne}

\subsection{Wygoda}

Jako zamiennik programu GNUMake system powinien obsługiwać taką samą składnię plików Makefile jak oryginał i udostępniać taki sam interfejs użytkownika, format komunikatów wypisywanych przez program będzie zachowywał zgodność z pierwowzorem tam, gdzie to możliwe, z racji rozszerzonej funkcjonalności.
Jedyną wspieraną wersją językową będzie wersja angielska.

\subsection{Bezpieczeństwo}

W zależności od konfiguracji, można opisywać dwa przypadki użycia systemu, które istotnie różnią się wymaganiami odnośnie bezpieczeństwa.

\subsubsection{Zaufana sieć (lokalna)}

W przypadku wykorzystania systemu do budowy oprogramowania na zdalnych maszynach, znajdujących się w zaufanym środowisku, np. tej samej sieci lokalnej, nie wymaga się szyfrowania połączeń pomiędzy jednostką koordynującą a jednostkami pomocniczymi.

\subsubsection{Niezaufane medium pośredniczące}

W przypadku wykorzystania do komunikacji pomiędzy maszynami np. Internetu, wymagane jest pełne szyfrowanie, protokół zdalnego wykonywania zadań oraz wymiany plików źródłowych i wynikowych powinien zapewniać przynajmniej taką poufność i potwierdzenie tożsamości, jak protokół SSHv2.

\subsection{Szybkość}

Docelowo system ma przyspieszać kompilację projektów, które składają się z wielu niezależnych części i których kompilacja może zostać zrównoleglona w wystarczającym stopniu aby zniwelować opóźnienia spowodowane zdalną komunikacją maszyn uczestniczących w procesie kompilacji.
Jeśli zaniedbać opóźnienia związane ze zdalną komunikacja przez sieć, wykonanie kompilacji przy pomocy systemu ma trać nie więcej niż $120\%$ czasu potrzebnego na wykonanie kompilacji przy pomocy programu GNUMake na pojedynczym komputerze.

\subsection{Użyteczność}

Z racji zamierzonej zgodności ze standardem (jakim są pochodne make’a) w większości wypadków nie będą potrzebne żadne modyfikacje do istniejących skryptów, obsługa programu nie będzie również wymagała zaznajamiania się z nowymi opcjami, zamierzamy implementować podzbiór opcji programu GNUMake i dodatkowo flagi regulujące działanie systemu koordynowania zadań na zdalnych maszynach.
Korzystanie z pliku konfiguracyjnego klienta SSH do przechowywania opisu sposobu łączenia i uwierzytelniania z jednostkami pomocniczymi powoduje, że w wielu środowiskach konfiguracja systemu redukuje się do wybrania hostów opisanych w tym pliku jako jednostek pomocniczych.
Takie rozwiązanie ułatwia również testowanie połączeń i zarządzanie kluczami uwierzytelniającymi.

\subsection{Komunikaty o błędach}

Komunikaty o błędach mają być takie same jak w programie GNUMake, dodatkowo komunikaty o niepoprawnej konfiguracji maszyn kompilujących, braku łączności lub innych problemach związanych ze zdalnym wykonywaniem zadań będą wypisywanie w terminalu, w którym zostanie wykonany program.
Tak jak GNUMake, program będzie wypisywał wyjście wszystkich komend, które wykona celem zbudowania zawartości opisanej w pliku Makefile.

\subsection{Estetyka}

Z racji sposobu interakcji z użytkownikiem ten punkt nie stosuje się do opisywanego systemu.


\end{document}
