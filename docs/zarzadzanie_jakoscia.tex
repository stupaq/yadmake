%% IOP - plan zarządzania jakością
%

\documentclass[a4paper]{article}

\usepackage{polski}
\usepackage[utf8]{inputenc}
\usepackage{fullpage}
\usepackage{indentfirst}
\usepackage{graphicx}
\usepackage{ifthen}
\usepackage{amssymb}
\usepackage{amsthm}
\usepackage{wrapfig}
\usepackage{siunitx}
\usepackage{xstring}
\usepackage{float}


\title{Plan zarządzania jakością projektu ,,System rozproszonej kompilacji''}
\author{Marta Drozdek, Anna Lewicka, Wacław Banasik, Mateusz Machalica}
\date{\today}

\begin{document}

\maketitle

\begin{table}[!h]
	\centering
	\begin{tabular}{|c|c|c|p{9cm}|}
		\hline
		\textbf{Data} & \textbf{Wersja} & \textbf{Autorzy} & \textbf{Opis zmian} \\ \hline
		25/03/2012 & 1 & Mateusz Machalica & Wstępna wersja dokumentu \\ \hline
		25/05/2012 & 2 & Mateusz Machalica & Dodano wstęp, opis wymagań dotyczących modularności i dokumentacji kodu \\ \hline
	\end{tabular}
\end{table}

\section{Wstęp}

Celem tego dokumentu jest ustalenie standardów zarządzania jakością gotowego systemu, kodu źródłowego oraz dokumentacji i sposobów sprawdzania zgodności systemu z wymaganiami funkcjonalnymi i pozafunkcjonalnymi.

\section{Kod źródłowy}

\subsection{Formatowanie i nazewnictwo}

Dla zachowania czytelności kodu źródłowego przed każdym wydaniem wszystkie pliki źródłowe zostaną przefiltrowane programem \verb+astyle+ aby ujednolicić formatowanie. Nie planujemy narzucać konkretnych standardów formatowania dla gałęzi, w których będzie znajdował się eksperymentalny, niezintegrowany z resztą programu kod.

Nazewnictwo identyfikatorów powinno być zgodne ze standardem wypracowanym przez Google, a opisanym w dokumencie ,,Google C++ Style Guide’’ w wersji 3.199 \cite{googstyle}.
Nie narzuca się powyższej konwencji na kod zamieszczany w gałęziach eksperymentalnych, tworzonych na użytek rozwijania fragmentów funkcjonalności, które nie zostały jeszcze zintegrowane z resztą kodu, jednak w obowiązkiem opiekunów gałęzi jest dostosowanie się do konwencji nazewniczych jeszcze przed łączeniem z istniejącym kodem.

Do weryfikacji zgodności kodu z przyjętymi standardami przed każdym wydaniem kod musi zostać sprawdzony skryptem \verb+cpplint+ \cite{cpplint}.

\subsection{Wyjątki i zgłaszanie błędów}

Jedyną formą zgłaszania błędów przez funkcje/metody są wyjątki. Nie dopuszczamy ustawiania pewnych globalnych flag ani zwracania wartości NULL, a wszystkie błędy zgłaszane przez wywoływane funkcje systemowe lub inne, nie obsługujące zgłaszania wyjątków, powinny być przekazywane dalej również w formie wyjątków.

Wyjątki, których wystąpienie jest związane z działaniem danej funkcji, a które nie zostały zadeklarowane w pliku nagłówkowym biblioteki standardowej \verb+stdexcept+ powinny być zadeklarowane w tym samym pliku nagłówkowym co funkcja, która może taki wyjątek podnieść. Ponadto wymaga się, aby każdy wyjątek dziedziczył z jednego z wyjątków zadeklarowanych w \verb+stdexcept+.

\subsection{Dokumentacja kodu}

W każdym module, każda funkcja, metoda i każda zmienna, z której nazwy jasno nie wynika, jakie dane przechowuje i która nie jest zmienną lokalną, powinna zostać opatrzona komentarzem z wyjaśnieniem jej przeznaczenia, który jest zgodny z formatem komentarzy dla systemu generującego automatyczną dokumentację dla kodu.

Każda funkcja i metoda, która wymaga dokumentacji, powinna deklarować w sposób zrozumiały dla systemu automatycznego generowania dokumentacji, czym jest wartość zwracana i co reprezentują parametry, jakie wyjątki mogą zostać podniesione podczas wywołania oraz jakie są efektu uboczne wywołania.

Każda klasa powinna zostać krótko opisana (jakie dane lub zasoby reprezentuje) w formacie zrozumiałym dla systemu generowania dokumentacji kodu.

Zaleca się również komentowanie nietrywialnych rozwiązań w kodzie, lecz z braku jednoznacznej miary nietrywialności reguła ta pozostaje jedynie zaleceniem.

Jako generatora dokumentacji używamy programu Doxygen.

\subsection{Modularność}

Musimy zagwarantować dużą modularność systemu, aby móc uzyskać cele założonych w wizji projektu, miedzy innymi możliwość działania w oparciu o różne protokoły komunikacyjne, oraz różne sposoby synchronizacji plików źródłowych i celów w zależności od dostępnej infrastruktury sieci łączącej komputery uczestniczące w procesie komunikacji.

Z tego powodu wymagamy, aby każdy moduł programu, który w opisie architektury systemu został sklasyfikowany, jako wymienialny w zależności od środowiska w jakim ma działać system, implementował interfejs (dziedziczył z klasy abstrakcyjnej), który ma zostać zaprojektowany jeszcze w fazie ustalania architektury systemu, nie podczas implementacji danego modułu.

Chcemy uniknąć sytuacji, w której implementujący dany moduł dostosuje ów interfejs do potrzeb własnego modułu, utrudniając lub wręcz uniemożliwiając późniejszą integrację modułów lub implementację modułów-zamienników.

Z tego powodu dopuszczamy we wczesnych iteracjach projektu, aby konkretne implementacje abstrakcyjnych interfejsów zawierały metody z niepełną implementacją, jeśli implementacja tych metod nie jest potrzebna w programie.

Każda metoda, która ma niepełną implementację (celowo nie implementuje swojej funkcjonalności) musi zaraz po wywołaniu podnosić wyjątek dziedziczący z \verb+std::runtime_error+ i zawierać w komentarzu słowa kluczowe ,,NOT IMPLEMENTED'', natomiast w komentarzu zawierającym opis metody dla generatora dokumentacji kodu musi znaleźć się odpowiednia informacja (w taki sposób, aby opis metody w wygenerowanej dokumentacji zawierał adnotację o niepełnej implementacji).

\section{Poprawność}

\subsection{Testy jednostkowe}

Każdy moduł powinien dostarczać zestaw testów jednostkowych weryfikujących (przynajmniej w podstawowym zakresie) poprawność. Pozwoli to na szybkie wykrywanie regresji.

Nie ustalono żadnego standardu ani nie narzucono frameworku testów, z racji małej docelowej liczby modułów nie wydaje się to potrzebnym a istotnie zwiększyłoby nakład pracy.

\subsection{Weryfikacja kodu}

Każdy fragment kodu przed połączeniem z istniejącym kodem powinien być zweryfikowany przez innego członka zespołu, niż autor. Weryfikacja powinna składać się z:
\begin{itemize}
\item zadeklarowanie przez autora zamierzonych efektów wykonania kodu, wartości zwracanych przy określonych danych wejściowych oraz wszelkich efektów ubocznych
\item następnie weryfikator sprawdza czy w jego opinii kod spełnia przedstawione wymagania
\item w razie wątpliwości powinna nastąpić dyskusja rozwiązania, jednak dopiero po próbie weryfikacji rozwiązania, chcemy uniknąć sytuacji kiedy weryfikator sugeruje się przekonaniem autora o poprawności kodu
\end{itemize}

\subsection{Zgłaszanie błędów}

Błędy zauważone poza powyżej opisanym procesem weryfikacji powinny być zgłaszane w systemie obsługi błędów dla projektu (aktualnie zintegrowany ze stroną projektu w serwisie GitHub). Jeśli błędy wykryte w procesie weryfikacji nie zostały natychmiast poprawione, albo ich naprawa rzutuje na inne moduły systemu, to informacja o takim błędzie również powinna zostać wprowadzona do systemu.

\subsection{Testy wdrożeniowe}

Każde wydanie programu, powinno być dodatkowo testowane poprzez zbudowanie własnego kodu źródłowego (jako system budowania dla projektu wykorzystujemy GNUMake’a).

\section{Miary produktowe}

\subsection{Wydajność}

Z racji eksperymentalnego charakteru systemu nie znamy kanonicznych sposobów testowania wydajności dla podobnych systemów. W początkowej fazie nie zamierzamy optymalizować fragmentów programu, lecz program jako całość poprzez wybór odpowiednich protokołów komunikacyjnych i architektury systemu.
Profilowanie poszczególnych fragmentów programu przewidujemy dopiero przed wydaniami stabilnymi.

Jedyną miarą wydajności powinno być porównanie z programem GNUMake. Zamierzamy skrócić czas wykonywania celów zawartych w Makefile’ach, dla których możliwe jest zrównoleglanie procesu kompilacji, tak aby wykorzystywał pełną moc obliczeniową ponad trzech standardowych komputerów domowych.

W przypadku budowy projektów, których kompilacja nie może być zrównoleglona w dostateczny sposób zamierzamy osiągnąć czas wykonania nie przekraczający 120\% czasu wykonania programu GNUMake dla tego samego wejścia.

\begin{thebibliography}{9}
\bibitem{googstyle} http://google-styleguide.googlecode.com/svn/trunk/cppguide.xml
\bibitem{cpplint} http://google-styleguide.googlecode.com/svn/trunk/cpplint/cpplint.py
\end{thebibliography}

\end{document}
